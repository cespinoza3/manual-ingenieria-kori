% !TeX spellcheck = es_ES
\section{Definiciones y especificación de requisitos de software}

% --- general del proyecto
\subsection{Definición general del proyecto de software}
\subsubsection{Propósito}
\subsubsection{Público objetivo y sugerencias de lectura}
\subsubsection{Alcance del producto}
\subsubsection{Perspectiva del producto}
\subsubsection{Clases y características del usuario}

% ---

% --- espectificacion de requerimientos
\subsection{Especificación de requerimientos del proyecto}
\subsubsection{Requisitos generales}

El sistema dispondrá de los siguientes módulos y submódulos mediante los cuales
se pondrá acceder a todas y cada una de las funcionalidades, las cuales serán 
descritas más adelante:

\begin{enumerate}
    \item 
\end{enumerate}

% !TeX spellcheck = es_ES
\subsubsection{Requisitos funcionales}

\newcommand{\funcionalidad}[7][none]{
	%\paragraph{#2}
	\item #2

	\begin{tabularx}{\textwidth}{|c|X|}
		\hline
		Módulo & #3 \\
		\hline
		Descripción & #4 \\
		\hline
		Entrada & #5 \\
		\hline
		Proceso & #6 \\
		\hline
		Salida & #7 \\
		\hline
	\end{tabularx}
}
\begin{enumerate}

	
	\funcionalidad{ejemplo}
	{modulo}
	{descripción}
	{entrada}
	{proceso}
	{salida}


\end{enumerate}
\subsubsection{Alcance del sistema}
% ---

% --- procedimientos de instalación y prueba
\subsection{Procedimientos de instalación y prueba}
\subsubsection{Procedimientos de desarrollo}

\subsubsection{Planificación}

% ---

% --- alcances tecnicos
\subsection{Alcances Técnicos}

\subsubsection{Requisitos no funcionales}
\subsubsection{Obtención e instalación}
\subsubsection{Especificaciones de prueba y ejecución}

% ---
